\documentclass{article}

\usepackage{listings}
\usepackage[utf8]{inputenc}
\usepackage[greek,english]{babel}
\usepackage{alphabeta}
% Set page size and margins
% Replace `letterpaper' with `a4paper' for UK/EU standard size
\usepackage[letterpaper,top=2cm,bottom=2cm,left=3cm,right=3cm,marginparwidth=1.75cm]{geometry}

% Useful packages
\usepackage{amsmath}
\usepackage{graphicx}
\usepackage[colorlinks=true, allcolors=blue]{hyperref}

% this can cause issues
\usepackage{float}

\title{Εργασία 1 Μικροεπεξεργαστές}
\author{Βογιατζής Χαρίσιος \\ ΑΕΜ:9192}

\lstset{
  language=C,                % choose the language of the code
  numbers=left,                   % where to put the line-numbers
  stepnumber=1,                   % the step between two line-numbers.        
  numbersep=5pt,                  % how far the line-numbers are from the code
  backgroundcolor=\color{white},  % choose the background color. You must add \usepackage{color}
  showspaces=false,               % show spaces adding particular underscores
  showstringspaces=false,         % underline spaces within strings
  showtabs=false,                 % show tabs within strings adding particular underscores
  tabsize=2,                      % sets default tabsize to 2 spaces
  captionpos=b,                   % sets the caption-position to bottom
  breaklines=true,                % sets automatic line breaking
  breakatwhitespace=true,         % sets if automatic breaks should only happen at whitespace
  title=\lstname,                 % show the filename of files included with \lstinputlisting;
}

\begin{document}

\maketitle

\href{https://github.com/charisvt/micro-lab1}{Github Source Code} \\

\section{Εισαγωγή}
Ο στόχος της εργασίας είναι να δημιουργήσουμε μία ρουτίνα σε ARM assembly η οποία θα δέχεται ένα αλφαριθμητικό και θα υπολογίζει ένα custom hash σύμφωνα με τις προδιαγραφές που δόθηκαν. \\

\section{hash.s}
Η συνάρτηση αυτή υλοποιεί το πρώτο κομμάτι της hashing διαδικασίας. \\
Αρχικά υπολογίζει υπολογίζει το μήκος του αλφαριθμητικού στον πρώτο βρόχο και θέτει έτσι την τιμή του hash. \\
Στο δεύτερο βρόχο προσθέτει μία κατάλληλη τιμή ανάλογα με τις ακόλουθες περιπτώσεις {κεφαλαίο, μικρό, ψηφίο, άλλο}. \\
Η συνάρτηση χρησιμοποιεί left shift για πολλαπλασιασμό με το 2, ένα lookup table και multiply and accumulate για τον υπολογισμό τετραγώνου αριθμού. \\
Επίσης δημιουργεί την global μεταβλητή hash\_result στην οποία αποθηκεύει την υπολογιζόμενη τιμή.\\
Οι δύο βρόχοι θα μπορούσαν να ενωθούν σε ένα χωρίς βλάβη του αλγορίθμου και με επιτάχυνση του, αλλά επιλέχθηκε να παραμείνουν χωριστοί για καλύτερη ανάγνωση και αντιστοιχία με τα ερωτήματα της εργασίας. \\
Ο αλγόριθμος αυτός είναι order-invariant, συνεπώς αλφαριθμητικά που αποτελούνται από τους ίδιους χαρακτήρες σε διαφορετικές θέσεις παράγουν το ίδιο hash. \\
Επίσης οι άλλοι χαρακτήρες δεν συνεισφέρουν στο hash (πέρα από την αλλαγή στο μήκος του αλφαριθμητικού).

\section{modhash.s}
Η συνάρτηση αυτή υλοποιεί το δέυτερο κομμάτι της hashing διαδικασίας. \\
Δουλεύει με δεδομένο το υπολογισμένο hash του προηγούμενου βήματος. \\
Εαν το hash είναι διψήφιο (>10) τότε προσθέτει τα ψηφία του hash και στη συνέχεια υπολογίζει το modulo 7. \\
Για τον υπολογισμόυ του αθροίσματος των ψηφίων ακολουθούμε την παρακάτω διαδικασία. \\
Αρχικά διαιρούμε με το 10 για να αποκόψουμε το τελευταίο ψηφίο. \\
Στη συνέχεια το πολλαπλασιάζουμε με το 10 και αφαιρούμε το αποτέλεσμα από τον αρχικό αριθμό με αποτέλεσμα να έχουμε απομονώσει μόνο το τελευταίο ψηφίο το οποίο προσθέτουμε στο άθροισμα. \\
Η διαδικασία αυτή εκτελείται επαναλληπτικά έως να επεξεργασθούν όλα τα ψηφία. \\
Στη συνέχεια με παρόμοιο τρόπο υπολογίζουμε το mod7, διαιρώντας δηλαδή αρχικά με το 7 και στην συνέχεια αφαιρώντας το αποτέλεσμα επί 7 από τον αρχικό αριθμό. \\
Τέλος, αποθηκεύει το αποτέλεσμα στη global hash\_result. \\

\section{fibonacci.s}
Η συνάρτηση αυτή υλοποιεί το τρίτο κομμάτι της hashing διαδικασίας. \\
Αποτελεί μια αναδρομική υλοποίηση του fibonacci. \\
Λόγω της αναδρομικής φύσης, στη συνάρτηση αυτή δόθηκε ιδιαίτερη έμφαση στη διαχείριση του stack και περιορίστηκε ή χρήση των non-scratch καταχωρητών στους R4 και R5 τους οποίους σώζουμε και επαναφέρουμε μεταξύ των αναδρομικών κλήσεων. \\
Δημιουργεί μία unitiliazed μεταβλητή fib\_result στο stack χρησιμοποιώντας το section .bss, δεσμεύοντας έτσι 4 bytes μόνο μία φορά και αποθηκεύει εκεί τα αποτελέσματα των υπολογισμών. \\

\section{crc.s}
Το προαιρετικό αυτό κομμάτι δημιουργεί ένα CRC-like checksum υπολογίζοντας το bitwise XOR όλων των χαρακτήρων του αλφαριθμητικού. \\
Αυτό το checksum είναι order sensitive και δεν αγνοεί τους άλλους χαρακτήρες όπως η προηγούμενη hash, το οποίο επιβεβαιώσαμε και στο testing. \\

\section{Testing}
Για τον έλεγχο της ορθής λειτουργίας των παραπάνω συναρτήσεων έχουν δημιουργηθεί οι αντίστοιχες reference συναρτήσεις σε C καθώς και οι testing συναρτήσεις που συγκρίνουν τα αποτελέσματα των 2 υλοποιήσεων. \\
Επιπλέον έχουν επιλεγεί κάποια χαρακτηριστικά αλφαριθμητικά για την επίδειξη των ιδιοτήτων του κάθε αλγορίθμου. \\

\section{UART}
Έχει υλοποιηθεί μία συνάρτηση ώστε να λαμβάνεται το αλφαριθμητικό μέσω UART (με χρήση της uart\_rx) αλλά δεν έχει δοκιμαστεί. \\

\section{Debugging}
Ένα πρόβλημα που αντιμετώπισαμε ήταν στην αναδρομική fibonacci την οποία κατά λάθος καλούσαμε με το hash από το πρώτο step και όχι μετά το modhash με αποτέλεσμα να προσπαθεί να υπολογίσει το fibonacci(164) το οποίο οδηγούσε σε memory error καθώς γέμμιζε το stack. \\
Για τον εντοπισμό και την επίλυση των προβλημάτων χρησιμοποιήθηκε ο debugger του Keil με breakpoints, step-by-step analysis και κοιτάζοντας τις τιμές των καταχωρητών. \\

\section{Source Code}
\href{https://github.com/charisvt/micro-lab1}{Github Source Code} \\

\end{document}
